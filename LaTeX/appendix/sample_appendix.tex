%! TEX root = ../thesis_main_sample.tex

\chapter{製本について}
	\label{app:seihon}

	修論・博論は「くるみ」製本以上の水準で製本するようにしてください。
	製本機での簡易の製本は、長期の保存に耐えないため、\textbf{禁止}とします。
	本文を自分で印刷し、下記に挙げるような店舗に持ち込むか、USBメモリ等にPDFを入れて店舗で直接印刷するとよいです。

	\begin{itemize}
		\item キンコーズ\footurl{http://www.kinkos.co.jp}
		\item コピーイン\footurl{http://www.copyinhongo.com}
	\end{itemize}

	また、PDFデータをオンライン入稿し、製本を送付してもらえるサービスもあります。

	いずれにせよ、入稿から製本までにかかる時間をよく考慮して、提出期間に間に合うように注意してください。

\chapter{研究遂行上の注意}\label{app:ethic}

	\section{研究倫理}
		\label{sec:echic}

		東京大学大学院教育学研究科が発行している『信頼される論文を書くために 第3版』
		\footurl{http://www.p.u-tokyo.ac.jp/~edudaiga/sonota/manural_march2017.pdf}
		をよく読み、研究及び論文執筆を行うこと。
		この冊子に書かれた要件が順守できていない学位論文は当然ながら受理されず,当該執筆者は修了に値しません。

	\section{インタビュー・質問紙調査}
		\label{sec:interview}

		インタビューや質問紙調査、その他生身の人間を対象とする研究においては、「ヒトを対象とした実験研究および調査研究に関する倫理審査委員会」の規程\footnote{世界医師会の「ヘルシンキ宣言」に準ずる}に従って倫理的な方法を用いるようにしてください。
		そして、インタビューや質問紙調査の対象となった個人や組織は論文中では匿名としてください。
		また、インタビューの書き起こしについては、インタビュー相手の了承を得て実施してください。

		個人情報を調査の過程で入手する研究では、準拠した個人情報保護ガイドラインおよび個人情報提供者の承諾を得ている旨を論文中に明記してください(研究方法の章など)。
		% いかなる調査においてもその過程で入手した個人情報を適切に保護することを心がけてください。
		個人情報保護法により \textquote{大学その他の学術研究を目的とする機関若しくは団体又はそれらに属する者} が \textquote{学術研究の用に供する目的} で個人情報を取り扱う場合、\textquote{必要な措置を自ら講じ、かつ、当該措置の内容を公表するよう努めなければならない} と定められています(同法第66条第3項)。
		例えば、日本教育学会の個人情報保護ガイドライン\footurl{http://www.jera.jp/outline/privacy_g/}などが参考になります。


\chapter{参考文献管理}
	\label{app:bibliography}

	Mendeley と GaCos の使い方を知っておくとよいでしょう。

\chapter{論文執筆に役立つ資料}
	\label{app:useful}

	準備中
	% 理系の〜 岡田先生推薦 方法論のリスト
